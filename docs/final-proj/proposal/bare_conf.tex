
%% bare_conf.tex
%% V1.4b
%% 2015/08/26
%% by Michael Shell
%% See:
%% http://www.michaelshell.org/
%% for current contact information.
%%
%% This is a skeleton file demonstrating the use of IEEEtran.cls
%% (requires IEEEtran.cls version 1.8b or later) with an IEEE
%% conference paper.
%%
%% Support sites:
%% http://www.michaelshell.org/tex/ieeetran/
%% http://www.ctan.org/pkg/ieeetran
%% and
%% http://www.ieee.org/

%%*************************************************************************
%% Legal Notice:
%% This code is offered as-is without any warranty either expressed or
%% implied; without even the implied warranty of MERCHANTABILITY or
%% FITNESS FOR A PARTICULAR PURPOSE! 
%% User assumes all risk.
%% In no event shall the IEEE or any contributor to this code be liable for
%% any damages or losses, including, but not limited to, incidental,
%% consequential, or any other damages, resulting from the use or misuse
%% of any information contained here.
%%
%% All comments are the opinions of their respective authors and are not
%% necessarily endorsed by the IEEE.
%%
%% This work is distributed under the LaTeX Project Public License (LPPL)
%% ( http://www.latex-project.org/ ) version 1.3, and may be freely used,
%% distributed and modified. A copy of the LPPL, version 1.3, is included
%% in the base LaTeX documentation of all distributions of LaTeX released
%% 2003/12/01 or later.
%% Retain all contribution notices and credits.
%% ** Modified files should be clearly indicated as such, including  **
%% ** renaming them and changing author support contact information. **
%%*************************************************************************


% *** Authors should verify (and, if needed, correct) their LaTeX system  ***
% *** with the testflow diagnostic prior to trusting their LaTeX platform ***
% *** with production work. The IEEE's font choices and paper sizes can   ***
% *** trigger bugs that do not appear when using other class files.       ***                          ***
% The testflow support page is at:
% http://www.michaelshell.org/tex/testflow/



\documentclass[conference]{IEEEtran}
% Some Computer Society conferences also require the compsoc mode option,
% but others use the standard conference format.
%
% If IEEEtran.cls has not been installed into the LaTeX system files,
% manually specify the path to it like:
% \documentclass[conference]{../sty/IEEEtran}





% Some very useful LaTeX packages include:
% (uncomment the ones you want to load)


% *** MISC UTILITY PACKAGES ***
%
%\usepackage{ifpdf}
% Heiko Oberdiek's ifpdf.sty is very useful if you need conditional
% compilation based on whether the output is pdf or dvi.
% usage:
% \ifpdf
%   % pdf code
% \else
%   % dvi code
% \fi
% The latest version of ifpdf.sty can be obtained from:
% http://www.ctan.org/pkg/ifpdf
% Also, note that IEEEtran.cls V1.7 and later provides a builtin
% \ifCLASSINFOpdf conditional that works the same way.
% When switching from latex to pdflatex and vice-versa, the compiler may
% have to be run twice to clear warning/error messages.






% *** CITATION PACKAGES ***
%
%\usepackage{cite}
% cite.sty was written by Donald Arseneau
% V1.6 and later of IEEEtran pre-defines the format of the cite.sty package
% \cite{} output to follow that of the IEEE. Loading the cite package will
% result in citation numbers being automatically sorted and properly
% "compressed/ranged". e.g., [1], \cite{seznec2007tage}, \cite{calder1997evidence}, \cite{lee1995branch}, \cite{jimenez2001dynamic}, \cite{jimenez2003fast} without using
% cite.sty will become [1], \cite{calder1997evidence}, \cite{jimenez2001dynamic}--\cite{lee1995branch}, \cite{seznec2007tage} using cite.sty. cite.sty's
% \cite will automatically add leading space, if needed. Use cite.sty's
% noadjust option (cite.sty V3.8 and later) if you want to turn this off
% such as if a citation ever needs to be enclosed in parenthesis.
% cite.sty is already installed on most LaTeX systems. Be sure and use
% version 5.0 (2009-03-20) and later if using hyperref.sty.
% The latest version can be obtained at:
% http://www.ctan.org/pkg/cite
% The documentation is contained in the cite.sty file itself.






% *** GRAPHICS RELATED PACKAGES ***
%
\ifCLASSINFOpdf
  % \usepackage[pdftex]{graphicx}
  % declare the path(s) where your graphic files are
  % \graphicspath{{../pdf/}{../jpeg/}}
  % and their extensions so you won't have to specify these with
  % every instance of \includegraphics
  % \DeclareGraphicsExtensions{.pdf,.jpeg,.png}
\else
  % or other class option (dvipsone, dvipdf, if not using dvips). graphicx
  % will default to the driver specified in the system graphics.cfg if no
  % driver is specified.
  % \usepackage[dvips]{graphicx}
  % declare the path(s) where your graphic files are
  % \graphicspath{{../eps/}}
  % and their extensions so you won't have to specify these with
  % every instance of \includegraphics
  % \DeclareGraphicsExtensions{.eps}
\fi
\usepackage{algorithm}
\usepackage{algpseudocode}
\usepackage{amsmath}
\usepackage{amsmath,amssymb,amsthm,latexsym,paralist, booktabs}
\usepackage{url}
\usepackage[pdftex]{graphicx}
% default pic path
\usepackage{bm}
\usepackage{mathtools}
\let\oldvec\vec
\renewcommand{\vec}[1]{\oldvec{\mathit{#1}}}
\newcommand{\mat}[1]{\mathbf{#1}} % undergraduate algebra version
\newcommand{\parallelsum}{\mathbin{\!/\mkern-5mu/\!}}
\graphicspath{{pics/}}
\usepackage{upgreek}
%\newcommand{\mat}[1]{\bm{\mathit{#1}}}




% *** Do not adjust lengths that control margins, column widths, etc. ***
% *** Do not use packages that alter fonts (such as pslatex).         ***
% There should be no need to do such things with IEEEtran.cls V1.6 and later.
% (Unless specifically asked to do so by the journal or conference you plan
% to submit to, of course. )


% correct bad hyphenation here
\hyphenation{op-tical net-works semi-conduc-tor}


\begin{document}
%
% paper title
% Titles are generally capitalized except for words such as a, an, and, as,
% at, but, by, for, in, nor, of, on, or, the, to and up, which are usually
% not capitalized unless they are the first or last word of the title.
% Linebreaks \\ can be used within to get better formatting as desired.
% Do not put math or special symbols in the title.
\title{CSCE 643 Multi View Geometry CV\\
Final Project Proposal}


% author names and affiliations
% use a multiple column layout for up to three different
% affiliations
\author{\IEEEauthorblockN{Yukun Zeng}
\IEEEauthorblockA{Department of Computer Science and Engineering\\
Texas A\&M University\\
College Station, TX 77840\\
Email: yzeng@tamu.edu}}

% conference papers do not typically use \thanks and this command
% is locked out in conference mode. If really needed, such as for
% the acknowledgment of grants, issue a \IEEEoverridecommandlockouts
% after \documentclass

% for over three affiliations, or if they all won't fit within the width
% of the page, use this alternative format:
% 
%\author{\IEEEauthorblockN{Michael Shell\IEEEauthorrefmark{1},
%Homer Simpson\IEEEauthorrefmark{2},
%James Kirk\IEEEauthorrefmark{3}, 
%Montgomery Scott\IEEEauthorrefmark{3} and
%Eldon Tyrell\IEEEauthorrefmark{4}}
%\IEEEauthorblockA{\IEEEauthorrefmark{1}School of Electrical and Computer Engineering\\
%Georgia Institute of Technology,
%Atlanta, Georgia 30332--0250\\ Email: see http://www.michaelshell.org/contact.html}
%\IEEEauthorblockA{\IEEEauthorrefmark{2}Twentieth Century Fox, Springfield, USA\\
%Email: homer@thesimpsons.com}
%\IEEEauthorblockA{\IEEEauthorrefmark{3}Starfleet Academy, San Francisco, California 96678-2391\\
%Telephone: (800) 555--1212, Fax: (888) 555--1212}
%\IEEEauthorblockA{\IEEEauthorrefmark{4}Tyrell Inc., 123 Replicant Street, Los Angeles, California 90210--4321}}




% use for special paper notices
%\IEEEspecialpapernotice{(Invited Paper)}




% make the title area
\maketitle

% As a general rule, do not put math, special symbols or citations
% in the abstract

% no keywords




% For peer review papers, you can put extra information on the cover
% page as needed:
% \ifCLASSOPTIONpeerreview
% \begin{center} \bfseries EDICS Category: 3-BBND \end{center}
% \fi
%
% For peerreview papers, this IEEEtran command inserts a page break and
% creates the second title. It will be ignored for other modes.
\IEEEpeerreviewmaketitle

\section{Introduction of 3D Reconstruction}
\subsection{Definition}
The essence of an image or a photo is a projection from 3D scene onto a 2D plane\cite{3d_recon_wiki}. Naturally we can think of that the depths of things are lost in this process as the final image we get from the projection is in 2D, thereby a lot of information is lost during the projection. 3D reconstruction,  the reverse process of obtaining 2D images from 3D scenes, is the creation of three-dimensional models from a set of images.
 
\subsection{Motivation}
The applications of 3D reconstruction are penetrated in our daily life. As 2-D imaging has problems of anatomy overlapping with each other and don't disclose the abnormalities, 3-D imaging can be used for both diagnostic and therapeutic purposes. The 3-D models we obtain from reconstruction can be used for operation planning, morphometric studies and has more reliability in orthopedics. Another recent tech trend is the development of 3-D printing, which also requires the model reconstruction to provide input to printers.

\section{3D Reconstruction from Uncalibrated Images}
As we mentioned above, due to the lost of information, from a single 2D image, no depth can be computed without additional information. However, this problem becomes feasible when we have multiple images (hopefully focusing on certain region). The classic process is composed of two major steps:
\begin{itemize}
	\item Match features across multiple images.
	\item Compute depth information using standard triangulation based on the above feature correspondence.
\end{itemize}
However, traditional approach like this requires careful calibration, which is a major drawback as it's error-sensitive. There are also other methods proposed but they either requires more than commonly known information or lead to certain drawbacks in results. Therefore, a relative 3D reconstruction approach using uncalibrated images is proposed in \cite{mohr1995relative}. The major advantage of this new approach over other previous ones are two fold:
\begin{itemize}
	\item It introduces the uncertain 3D information only at the latest stage of the perception process, and only if it is needed.
	\item It avoids the unstable calibration process, and therefore relies only on the accuracy of the measures in the image during the processing.
\end{itemize}
The basic idea of our approach for implementation is that it incorporate calibration and reconstruction in the same optimization process by an implementation of parameters estimation theory using Levenberg-Marquardt algorithm. Different from other earlier methods, our reference approach is formulated globally as a least squares estimation method that does not need to first estimate the epipolar geometry and it makes full use of redundancies in multiple images. Throughout this project, we will propose an implementation of the relative 3D reconstruction approach and evaluate the method using reasonable experiments to show its performance.

\bibliographystyle{IEEEtran}
% argument is your BibTeX string definitions and bibliography database(s)
\bibliography{refs}

\end{document}


